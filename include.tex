\subsection{Include}
This subsection also comes from a different file, added to the main document 
with the include command. This is the more complex command of the two: 

\begin{itemize}
	\item It forces a page break (clearpage) before inserting the new content;
	\item Ideal for for bigger contents, like entire book chapters;
	\item Cannot be nested;
	\item It creates a dedicated .aux file, meaning the file could potentially 
	compiled on its own (this one cannot though!) - the \texttt{subfile} and 
	\texttt{standalone} packages offer a better alternative if you want to do 
	this; % to be fixed
	\item Can only be used in the document body, not in preamble, packages or 
	restricted/math mode;
	\item Also here, numbering remains consistent.
\end{itemize}

The command \verb+\includeonly{filename1,filename2,...}+ can be used in the 
preamble of the main document to only load selected sections. This allows you 
to only work on one section at a time while also keeping consistent numbering 
and referencing.