\documentclass[12pt, letterpaper]{article} % defines the type (class) of the document, takes [extra parameters, separated by commas] like font (def = 10) and paper size (e.g. [12pt, letterpaper])

\usepackage[utf8]{inputenc} % encoding for the document, can be omitted or changed but utf-8 is best

\usepackage{graphicx} % package used to add pictures to document
\graphicspath{{images/}{more images/}} % default directories for images

\usepackage{geometry} % package used to change length & layout of paper size, margins, footnote, header, orientation, etc.
\geometry{a4paper, portrait, margin = 1in} % paper size, orientation, margin size

\usepackage{multicol} % allows to have multiple columns
\setlength{\columnsep}{1cm} % sets columns spacing

\usepackage[dvipsnames]{xcolor} % allows use of colour in document + extra names option
\pagecolor{white} % sets page background colour, reversed by \nopagecolor 
\color{black} % sets default text colour

\definecolor{mypink1}{rgb}{0.858, 0.188, 0.478} % four ways to define custom colours
\definecolor{mypink2}{RGB}{219, 48, 122}
\definecolor{mypink3}{cmyk}{0, 0.7808, 0.4429, 0.1412}
\definecolor{mygray}{gray}{0.6}

\colorlet{LightRubineRed}{RubineRed!70!} % three ways to customize colours (xcolor only)
\colorlet{OrangeGreen}{green!10!orange!90!}
\definecolor{HTMLColor}{HTML}{00F9DE}

\setlength{\parindent}{20pt} % sets the size of the indent
\setlength{\parskip}{1ex} % sets the paragraph spacing
\renewcommand{\baselinestretch}{1.15} % sets interline spacing
\pagenumbering{arabic} % sets the style of page numbering, r(R)oman and a(A)lph are alts

\title{\LaTeX cheatsheet} % title of the document

\author{frostedmist\thanks{so done with the world} , too much tea} % author of the document + author affiliations (thanks). 

\date{April 2019} % date, can be substituted by the command \today

\begin{document} % start of the document body

\maketitle % inserts the preamble info (title/author/date) in the document body

\section{Font style and size}

\subsection*{Here is a series of font styles:}

This is \textbf{in bold!} 
This is \textit{in italics!}
This is \underline{underlined!}
This is \textbf{\textit{\underline{a combination!}}}
This is \textsl{slanted!}
This is \textsc{in small caps!}
This is \emph{emphasized depending on the context!} % can be altered by packages

\subsection*{Here are the default font families:}

\textrm{This is Serif (roman)!}
\textsf{This is Sans Serif!}
\texttt{This is Typewriter (monospace)!}

\subsection*{Here are some font sizes (relative to starting size):} % list not exhaustive

\tiny{This is tiny!}
\small{This is small!}
\large{This is large!}
\Large{I said Large!}
\LARGE{NO, LARGE!!!}
\huge{There we go.}
\normalsize{Back to normal now.}

\clearpage % inserts page break, use before the start of a new section
% \newpage is an alt that breaks the page exactly at that point + starts a new column (if using)

\section{Formatting}

\begin{abstract}
	This is how to format an abstract.
\end{abstract}

This is a first paragraph.\\
This is a line break.\newline
This is also a line break. % basic ways, list not exhaustive

\begin{center}
	This centres the text! You can also align it left (flushleft) or right (flushright), or do nothing to leave it justified.
	
	A blank line creates a new paragraph!\par
	The par command also creates a new paragraph!
\end{center}

You can insert horizontal space in text! Like this \hspace{1cm} here (all units allowed.) You can also have a break that \hfill automatically fills all the space available. \\ You can make this pretty by adding a line \hrulefill or dots \dotfill like this. \\Useful for signatures and indices!

You can also insert vertical space! Again, you can either \vspace{5mm}\\specify a distance or let it automatically \vfill fill the space available (will take into account other elements in page). Smallskip, medskip and bigskip are also somewhat dynamic ways to insert vertical space.

\begin{multicols}{3} % call multicols* for unbalanced columns
	[\subsection*{Columns}
	This is header text on top of the columns. Put whatever you want here but figures/tables.]
	This text should end up in column one. \\ This text should end up in column 2. \\ This text should end up in column 3.
\end{multicols}
And would you look at that - it did. Yay.


\clearpage
\subsection*{Sections hierarchy (commands):} % package titlesec allows customisation

\begin{enumerate}
	\setcounter{enumi}{-2} 
	\item part: only available in report doc class
	\item chapter: only available in book doc class
	\item section
	\item subsection
	\item subsubsection
	\item paragraph
	\item subparagraph
\end{enumerate}
Any of these can be made unnumbered by adding an * before the opening curly brace.

\begin{table}[!h]
	\centering
	\begin{tabular}{|l|l|}
		\hline
		Abbreviation & Value                                      \\ \hline
		pt           & $\sim$0.0138 inch or 0.3515 mm             \\ \hline
		mm           & a millimeter                               \\ \hline
		cm           & a centimeter                               \\ \hline
		in           & an inch                                    \\ \hline
		ex           & $\sim$height of an 'x' in the current font \\ \hline
		em           & $\sim$width of an 'M' in the current font  \\ \hline
		mu           & math unit equal to 1/18 em                 \\ \hline
	\end{tabular}
	\caption{LaTeX units summary}
	\label{table:units}
\end{table}

\subsection*{Paragraph formatting}

By default, the first paragraph of a section or a chapter is not indented.

The second is, and the indent size is defined in the preamble.

All subsequent ones are too.

\noindent The noindent command changes this.\par

This is a paragraph that spans multiple lines that I am using to test the line spacing options. Really there is not that much to see here. Carry on.

\clearpage
\section{Figures and referencing} \label{sec:fig} % this labels the section for referencing

\begin{figure}[h] % inserts figure + caption, label and reference [h]ere
    \centering % alignment
    \includegraphics[width=0.9\textwidth]{scream.jpg} % inserts image, with a width of 90% of the text width
    \caption{me} % inserts caption, used in list of figures
    \label{fig:scream} % numbers and labels image for reference
\end{figure}

This will refer to the figure label, Figure \ref{fig:scream}, while this will refer to the page it is in, Page \pageref{fig:scream}.

You can also refer to a specific section, like Section \ref{sec:fig} in which this is.

\clearpage
\section{Lists and maths}
Environments are sections of the document that present themselves in a different way to the rest. These are some of them.

\begin{itemize}
  \item This is an unordered list
  \item It uses bullet points
  \item Text can be of any length
\end{itemize}

\begin{enumerate}
% \setcounter{enumi}{3} <- start at a different number
  \item This is an ordered list
  \item It uses numbers
  \item The list number increases with each item
\end{enumerate}

These are ways to write mathematical expressions, inline mode:
$E=mc^2$, 
\(E=mc^2\), 
\begin{math} E=mc^2 \end{math}

These are ways to write mathematical expressions, display mode:

\begin{equation} \label{eq:1} % numbered
E=mc^2
\end{equation}

\[E=mc^2\] % unnumbered

\begin{displaymath} % unnumbered alt
E=mc^2
\end{displaymath}

% do NOT use $$ E=mc^2 $$ 

Like everything else, display mode mathematical equations can be referenced, e.g. equation \ref{eq:1}. Unnumbered equations can also be referenced by number but cannot be easily recognised.

\clearpage
\section{Tables} % use tablesgenerator.com to convert spreadsheets to code (e.g. table 1)

\begin{table}[h] % inserts table [h]ere
	\centering % centers the table in page
	\begin{tabular}{|| l | c | r ||} % defines layout, number/align of columns & vertical lines
		\hline % line at the top of first row
		Col1 &	Col2 &	Col2	\\ \hline\hline % & breaks a cell, \\ breaks a row
		1	 &	6	 &	87837	\\ \hline
		2	 &	7	 &	78		\\ \hline
		5	 &	88	 &	788 	\\ [1ex] \hline % [ex] adds vertical space to row
	\end{tabular}
	\caption{Fancy table} % label and caption go after the table
	\label{table:data} % numbers and labels the table for reference
\end{table}

The table can then be referred to as Table \ref{table:data} on Page \pageref{table:data}.

\section{Colours}

The \texttt{xcolour} package allows to colour things. The basic colours it supports are \textcolor{white}{white}, \textcolor{black}{(white), black}, \textcolor{red}{red}, \textcolor{green}{green}, \textcolor{blue}{blue}, \textcolor{cyan}{cyan}, \textcolor{magenta}{magenta}, \textcolor{yellow}{and yellow}.

Adding \texttt{dvipsnames} to the package in the preamble allows you to name a few more, like \textcolor{TealBlue}{Teal Blue} or \textcolor{WildStrawberry}{Wild Strawberry} (careful with the caps!). You can also change \colorbox{BurntOrange}{the background colour} for the text.

The color command (instead of textcolor) can be used \color{cyan}to switch the colour of an entire block of text until it ends - or until the end of the environment. Remember to \color{black} switch back in that case!

Here is a list of custom colours (defined in the preamble) instead.
\begin{itemize}
	\item \textcolor{mypink1}{Pink with rgb}
	\item \textcolor{mypink2}{Pink with RGB}
	\item \textcolor{mypink3}{Pink with cmyk}
	\item \textcolor{mygray}{Gray with gray}
	\item \textcolor{LightRubineRed}{Rubine red at 70\% intensity} % print % by adding a \
	\item \textcolor{OrangeGreen}{A mix of 10\% green and 90\% orange}
	\item \textcolor{HTMLColor}{Defined with HTML code}
\end{itemize}

These can be used for any element that takes a colour as parameter, like a line.

\noindent {\color{TealBlue} \rule{\linewidth}{1mm}} % draws a line

\clearpage
\section{Table of contents}

\tableofcontents

% tocs don't include unnumbered sections, if one needs to be added, then the command \addcontentsline{toc}{section}{Unnumbered Section} can be used above that unnumbered section

\addcontentsline{toc}{section}{Include me please!}
\section*{Include me please!}

This unnumbered section is in the toc.

\section*{I want to be left out!}

This unnumbered section is not in the toc.

\end{document} % end of the document body
